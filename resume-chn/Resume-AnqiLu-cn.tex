%%%%%%%%%%%%%%%%%%%%%%%%%%%%%%%%%%%%%%%%%
% Friggeri Resume/CV
% XeLaTeX Template
% Version 1.1 (9/2/15)
%
% This template has been downloaded from:
% http://www.LaTeXTemplates.com
%
% Original author:
% Adrien Friggeri (adrien@friggeri.net)
% https://github.com/afriggeri/CV
%
% License:
% CC BY-NC-SA 3.0 (http://creativecommons.org/licenses/by-nc-sa/3.0/)
%
% Important notes:
% This template needs to be compiled with XeLaTeX and the bibliography, if used,
% needs to be compiled with biber rather than bibtex.
%
%%%%%%%%%%%%%%%%%%%%%%%%%%%%%%%%%%%%%%%%%

\documentclass[]{friggeri-cv-cn} % Add 'print' as an option into the square bracket to remove colors from this template for printing


\addbibresource{bibliography.bib} % Specify the bibliography file to include publications

\begin{document}

\header{卢}{安琪}{全栈软件工程师} % Your name and current job title/field

%----------------------------------------------------------------------------------------
%	SIDEBAR SECTION
%----------------------------------------------------------------------------------------

\begin{aside} % In the aside, each new line forces a line break
\section{联系方式}
上海市
~
+86 138 1641 2974
~
\href{mailto:marine.luanqi@gmail.com}{marine.luanqi@gmail.com}
~
\href{https://www.linkedin.com/in/anqilu}{linkedin.com/in/anqilu}
\section{语言}
中文 母语
英语 IELTS 7.5
\section{programming languages}
{\color{red} $\varheartsuit$} Python
JavaScript
GO, JAVA
\section{web frameworks}
Flask
React (Redux)
\section{databases}
MySQL
MongoDB
Redis
\section{ide}
PyCharm
WebStorm
\section{editors}
Vim
\end{aside}

%----------------------------------------------------------------------------------------
%	WORK EXPERIENCE SECTION
%----------------------------------------------------------------------------------------

\section{工作经验}

\begin{entrylist}
%------------------------------------------------
\entry
	{目前}
	{2017. 10}
	{数坤(北京)网络科技有限公司}
	{上海,中国}
	{\emph{全栈软件工程师}
		\begin{itemize}
			\item 参与冠脉诊断医疗软件完整的研发周期。该软件处理冠脉CT图像、分析数据、并为医生生成智能诊断报告。
			\item 参与开发一系列内部支持工具,包括图像标注平台与数据管理平台。
			\item 负责中间件与通用组件的开发(例如:任务队列,图像处理组件,标注组件)。
			\item 线上服务的日常运维。
		\end{itemize}
	}
%------------------------------------------------
\entry
	{2017.09}
	{2017.04}
	{饿了么}
	{上海,中国}
	{\emph{高级软件工程师}
		\begin{itemize}
			\item 独立设计、研发并首发判责服务。该服务并发流量高,处于整个订单业务的关键路径上。
			\item 负责订单终止服务(包括售中异常服务与退款服务)的新功能开发与技术改造。
		\end{itemize}
	}
%------------------------------------------------
\entry
	{2017.03}
	{2015.11}
	{IBM}
	{上海,中国}
	{\emph{软件工程师}
		\begin{itemize}
		   \item 参与RDO在PPC64架构上的解决方案。主要负责核心组件的移植,高可用性(HA)以及自动化部署OpenStack。
			\item 开发并维护PPIM (Pure Power Integrated Manager)。负责VirtualMachine管理模块。
			\item 参加POC项目Heimdallr,用于Power机器的性能监控,负责系统性能数据捕获与可视化。
		\end{itemize}
	}
%------------------------------------------------
\\
	\entry
	{2014.07}
	{2014.01}
	{EMC\textsuperscript{2}}
	{上海,中国}
	{\emph{软件工程师实习生}
		\begin{itemize}
			\item 基于CakePHP,开发与维护LMS2 (实验室管理系统II)。
			\item 开发网络应用,支持实验室用户的特定需求。
		\end{itemize}
	}
%------------------------------------------------
\\
	\entry
	{2013.12}
	{2013.01}
	{上海众言网络科技有限公司}
	{上海,中国}
	{\emph{软件工程师实习生}
		\begin{itemize}
			\item 基于Django,开发问卷调研系统。
			\item 使用NumPy和PANDAS,处理调研数据。
		\end{itemize}
	}
%------------------------------------------------
\end{entrylist}

%----------------------------------------------------------------------------------------
%	EDUCATION SECTION
%----------------------------------------------------------------------------------------

\section{教育背景}

\begin{entrylist}
\entry
{2015.11}
{2014.09}
{爱丁堡大学}
{爱丁堡, 英国}
{
	\emph{计算机科学硕士学位}
	\begin{itemize}
		\item 一等学位毕业生
		\item GPA: 3.75/4.0
		\item 主要课程:高级数据库;分布式系统;极限计算;信息论;Java编程;并行编程语言及系统;语义型网络系统;软件开发架构、流程及管理
	\end{itemize}
}
%------------------------------------------------
\entry
{2014.06}
{2010.09}
{上海财经大学}
{上海,中国}
{
	\emph{电子商务学士学位}\\
	GPA: 3.38/4.0
}
%------------------------------------------------
\end{entrylist}

%%----------------------------------------------------------------------------------------
%%	PUBLICATIONS SECTION
%%----------------------------------------------------------------------------------------
%
\section{发表专利}

\printbibsection{patent} % Print all patents from the bibliography

%----------------------------------------------------------------------------------------
\end{document}
